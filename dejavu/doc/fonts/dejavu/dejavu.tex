%% Copyright 2011 Pavel Farar
%
% This work may be distributed and/or modified under the
% conditions of the LaTeX Project Public License, either version 1.3
% of this license or (at your option) any later version.
% The latest version of this license is in
%   http://www.latex-project.org/lppl.txt
% and version 1.3 or later is part of all distributions of LaTeX
% version 2005/12/01 or later.
%
% This work has the LPPL maintenance status `maintained'.
% 
% The Current Maintainer of this work is Pavel Farar


\documentclass[oneside]{scrartcl}

\usepackage[utf8]{inputenc}
\usepackage[czech, polutonikogreek, english]{babel}
\usepackage[IL2, T1]{fontenc}

\usepackage{DejaVuSans}
\usepackage{DejaVuSansMono}
\usepackage{DejaVuSerif}

\usepackage{booktabs}

\usepackage{textcomp}
\usepackage[colorlinks]{hyperref}
\hypersetup{pdfauthor={Pavel Farar}}

\usepackage{graphicx}


\title{Support package for DejaVu fonts}
\author{Pavel Farář\\
\href{mailto:pavel.farar@centrum.cz}{pavel.farar@centrum.cz}}

\newcommand{\myBig}{\Huge}
\newcommand{\myExperimental}{\color{blue}}

\begin{document}

\selectlanguage{english}%

\maketitle

\tableofcontents


\section{Introduction}

This package contains the LaTeX support for the DejaVu fonts. These fonts
are derived from the Vera fonts and there are already other Vera derivatives
on CTAN (Bera and Arev). The main reason why I created support for the DejaVu
fonts is that I am not satisfied with the Bera and Arev fonts for the Czech
and Slovak languages. However, other people should also benefit from this
package---the fonts support more languages and they have several additional
styles.

The following figures show all supported styles. There are four basic and four
condensed styles for DejaVu Sans and  and DejaVu Serif. DejaVu Sans Mono has
four basic styles. DejaVu Sans ExtraLight is not supported at this moment, but
it will come some time later. The additional styles not present in the original
Vera fonts are in blue color and they are considered experimental.

% !!! Use slanted (not "it") in the following table
\begin{figure}[!htb]
\centering{%
  \begin{tabular}{ll}
    {\usefont{T1}{DejaVuSans-TLF}{m}{n}DejaVu Sans} &
    {\usefont{T1}{DejaVuSans-TLF}{b}{n}DejaVu Sans Bold} \\
    {\usefont{T1}{DejaVuSans-TLF}{m}{sl}DejaVu Sans Oblique} &
    {\usefont{T1}{DejaVuSans-TLF}{b}{sl}DejaVu Sans Bold Oblique} \\
    {\usefont{T1}{DejaVuSans-TLF}{c}{n}\myExperimental DejaVu Sans Condensed} &
    {\usefont{T1}{DejaVuSans-TLF}{bc}{n}\myExperimental DejaVu Sans Condensed
      Bold} \\
    {\usefont{T1}{DejaVuSans-TLF}{c}{sl}\myExperimental DejaVu Sans Condensed
      Oblique} &
    {\usefont{T1}{DejaVuSans-TLF}{bc}{sl}\myExperimental DejaVu Sans Condensed
      Bold Oblique} \\
    {\usefont{T1}{DejaVuSans-TLF}{el}{n}\myExperimental DejaVu Sans ExtraLight} &
       ~

  \end{tabular}
\caption{The styles of DejaVu Sans}}
\end{figure}

\begin{figure}[!htb]
\centering{%
  \begin{tabular}{ll}
    {\usefont{T1}{DejaVuSerif-TLF}{m}{n}DejaVu Serif} &
    {\usefont{T1}{DejaVuSerif-TLF}{b}{n}DejaVu Serif Bold} \\
    {\usefont{T1}{DejaVuSerif-TLF}{m}{it}\myExperimental DejaVu Serif Italic} &
    {\usefont{T1}{DejaVuSerif-TLF}{b}{it}\myExperimental DejaVu Serif Bold Italic} \\
    {\usefont{T1}{DejaVuSerif-TLF}{c}{n}\myExperimental DejaVu Serif Condensed} &
    {\usefont{T1}{DejaVuSerif-TLF}{bc}{n}\myExperimental DejaVu Serif Condensed Bold} \\
    {\usefont{T1}{DejaVuSerif-TLF}{c}{it}\myExperimental DejaVu Serif Condensed Italic} &
    {\usefont{T1}{DejaVuSerif-TLF}{bc}{it}\myExperimental DejaVu Serif Condensed Bold Italic}
  \end{tabular}
\caption{The styles of DejaVu Serif}}
\end{figure}

\begin{figure}[!htb]
\centering{%
  \begin{tabular}{ll}
    {\usefont{T1}{DejaVuSansMono-TLF}{m}{n}DejaVu Sans Mono} &
    {\usefont{T1}{DejaVuSansMono-TLF}{b}{n}DejaVu Sans Mono Bold} \\
    {\usefont{T1}{DejaVuSansMono-TLF}{m}{sl}DejaVu Sans Mono Oblique} &
    {\usefont{T1}{DejaVuSansMono-TLF}{b}{sl}DejaVu Sans Mono Bold Oblique}
  \end{tabular}
\caption{The styles of DejaVu Sans Mono}}
\end{figure}

The fonts have Latin, Cyrillic and Greek letters (and even more).
The fonts cover many languages, but the three
families do not have exactly the same characters. DejaVu Sans has usually more characters than the other fonts. Look at the file \texttt{languages.txt} what
languages are supported.

This package currently supports encodings T1, OT1, IL2, TS1, T2*, X2, QX
and LGR. The fonts are included in the original
TrueType format and in the converted Type 1 format.


\section{License}

The DejaVu fonts are derived from the Vera fonts and take some
characters from the Arev fonts. Therefore the Vera and Arev licenses
apply. DejaVu changes are in the public domain.

The fonts in TrueType format are original files and the
fonts in Type 1 format were converted by me
using FontForge.

All the support files are licensed under the \LaTeX\ Project Public License,
either version 1.3 of this license or (at  your option) any later version.


\section{Problems in Arev and Bera}

The fonts Bera and Arev do not have all accented letters precomposed.
Some letters are composed from the base letter and accent and some
results are really bad. I will show two quite common problems.

First, the letters dcaron, tcaron, lcaron and Lcaron should use a special type
of caron, not the apostrophe. And the letters should not be much wider
than the unaccented letters! Look at the following text using DejaVu Sans:\\
{\Huge\usefont{T1}{DejaVuSans-TLF}{m}{n}žluťoučká loďka}

Then compare it with the same text using Arev and Bera:\\
{\Huge
{\usefont{T1}{fav}{m}{n}žluťoučká loďka}\\	% arev sans
{\usefont{T1}{fvs}{m}{n}žluťoučká loďka}	% bera sans
}

Second, the original Vera fonts have different shapes of accents for capital and
small letters. Bera and Arev compose the missing accented letters with only
one type of accents---that used for small letters. This results in capital
letters with different shapes of accents. First, look at the accents in DejaVu
fonts:\\
{\Huge\usefont{T1}{DejaVuSans-TLF}{m}{n}PŘÍŠERNÝ příšerný}

Then compare the accents in the letters Rcaron and Scaron in Bera fonts:\\
{\Huge\usefont{T1}{fvs}{m}{n}PŘÍŠERNÝ příšerný}		% bera sans

Arev fonts do not have problems in exactly the same letters as Bera fonts,
but they have similar problems:\\
{\Huge\usefont{T1}{fav}{m}{n}PŘÍŠERNÝ příšerný}		% arev sans

Look at the accents in Iacute and Yacute.
They are different from the accents used in the small letters. On the other
hand, the accents in the letters Rcaron and Scaron are the same as those
used in the small letters. And you can see that the different height of
accents in the first three accented capital letters does not look very good.

The problems with accents in capital letters are not just Czech-specific, you
can see them also in other central European languages, such as Polish. And there
is also the non-ideal position of accents in letters like Ccaron and ccaron.


\section{The LGR encoding for Greek}

The capital Greek accented letters are composed from the accent and the
capital letter. There are (currently) no kerning pairs between these accents
and capital letters in the fonts. Therefore, I added additional kerning pairs
for these combinations and I think that the result is much better:
\textgreek{>`A <'O <O >'E >E}.

The combination accent and capital letter doesn't still give the same result
as the precomposed letter. The combination has bigger left side bearing than
the precomposed letter. I may do something with it later, but the most visible
problem for me was the position of accents.

The extra kerning pairs are not included in the fonts, but were used to
generate the tfm files and I put them to the package for the case you want
to look at them.


\section{Using the package}

The package was created using \texttt{fontools}, but with some changes. From
\texttt{fontools} also come the mixed case names and the suffix \texttt{-TLF}
(tabular lining figures) of the font families and also the mixed case
names of the packages for the individual font families.

There are two map files for the fonts: \texttt{dejavu-truetype.map} and \texttt{dejavu-type1.map}.
You should use just one of them. The file \texttt{dejavu-type1.map} is a safe
choice and you will get the full power of \TeX. If you just want to create PDF
files with pdf\TeX\ or pdf\LaTeX\ you can use the file
\texttt{dejavu-truetype.map} and the original TrueType fonts will be used.
You install the fonts with a command like:
\begin{verbatim}
updmap-sys --enable Map=dejavu-type1.map
\end{verbatim}

There are several packages that you can use to set the fonts. You
must explicitly use the package \texttt{fontenc} or \texttt{textcomp} (if
needed). The easiest way to use the fonts is the package \texttt{dejavu}:
\begin{verbatim}
\usepackage[T1]{fontenc}
\usepackage{dejavu}
\end{verbatim}

\sloppy
This sets the font \texttt{DejaVuSerif-TLF} as the default serif family,
the font \texttt{DejaVuSans-TLF} as the default sans serif family
and finally the
font \texttt{DejaVuSansMono-TLF} as the default typewriter family.

\sloppy
There are also several packages that set the individual font families.
They have the same name as the font family, but without the suffix:
\texttt{DejaVuSerif}, \texttt{DejaVuSerifCondensed},
\texttt{DejaVuSans}, \texttt{DejaVuSansCondensed} and \texttt{DejaVuSansMono}.

For example the package \texttt{DejaVuSerif} sets the font \texttt{DejaVuSerif-TLF}
and you use it this way:
\begin{verbatim}
\usepackage[T1]{fontenc}
\usepackage{DejaVuSerif}
\end{verbatim}

If you want to use some font together with another font that has somewhat
different proportions, you can use the option \texttt{scaled}:
\begin{verbatim}
\usepackage[scaled=0.9]{DejaVuSerif}
\end{verbatim}

You can also typeset some text in a desired font like this:
\begin{verbatim}
{\usefont{T1}{DejaVuSerifCondensed-TLF}{m}{it}Text in
DejaVu Serif Condensed Italic}
\end{verbatim}

You can see the available series/shape combinations in the
table~\ref{tableShape}. There are just the combinations that really
exist in the fonts, but you can use also some other combinations and
then the substitution will be used. The bold extended series
will change to the bold series and you can use the slanted or italic
shape for all families.

\begin{table}[!htb]
\centering{%
  \begin{tabular}{ll}
    \toprule
    font family & series/shape combinations \\
    \midrule
    DejaVuSerif-TLF & m/n, m/it, b/n, b/it, c/n, c/it, bc/n, bc/it \\
    DejaVuSerifCondensed-TLF & m/n, m/it, b/n, b/it \\
    DejaVuSans-TLF & m/n, m/sl, b/n, b/sl, c/n, c/sl, bc/n, bc/sl, el/n \\
    DejaVuSansCondensed-TLF & m/n, m/sl, b/n, b/sl \\
    DejaVuSansMono-TLF & m/n, m/sl, b/n, b/sl \\
    \bottomrule
  \end{tabular}
\caption{The series/shape combinations for the fonts}\label{tableShape}}
\end{table}





\section{Known bugs}

I put the letters \emph{Cyrillic letter short I with tail},
\emph{Cyrillic letter EL with tail} and \emph{Cyrillic
letter EM with tail} to the slots where the same letters with descender
should be. They are somewhat similar and some languages should use exactly
these letters. I think that it is better than leaving the slots empty.

I experienced problems with letters \emph{dcroat} and \emph{Idotaccent}.
The use of the fonts DejaVu-LGC did not help. Now
I use different names of characters in the encoding T1 for the TrueType and
Type 1 fonts and it works fine for me.

The encodings OT1 and IL2 are not exactly the same as implemented in Computer
Modern fonts, they are similar to the encodings in \TeX\ Gyre fonts. There
is no change from dollar to pound in the italic shape, but I believe that
it should not be a serious problem.

The substitution of two or three hyphens in the monospaced font is not ideal.
The behaviour is similar to both the Computer Modern and the \TeX\ Gyre fonts,
but you may get some warnings when you use the TrueType fonts and the text is
not in the \texttt{verbatim} environment. I recommend you to use one hyphen
for en-dash and the expression \texttt{\{-\}\{-\}} for em-dash.

Please \href{mailto:pavel.farar@centrum.cz}{send me} bug reports and
suggestions about this package.




\end{document}
